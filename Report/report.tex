% report.tex - Frontpage for the report (place in Report/)
\documentclass[a4paper,12pt]{article}
\usepackage[utf8]{inputenc}
\usepackage[T1]{fontenc}
\usepackage{graphicx}
\usepackage{eso-pic}
\usepackage[HTML]{xcolor}
\usepackage{tikz}
\usepackage{geometry}
\usepackage{parskip}
\usepackage{setspace}
\usepackage{caption}
\usepackage{subcaption}
\usepackage{float}
\usepackage{gensymb}
\usepackage{ulem}
\usepackage{amsmath}
\usepackage{amsfonts}
\usepackage{microtype}
\usepackage{geometry}
\usepackage{setspace}
\usepackage{tabularx}
\usepackage{longtable}
\usepackage{booktabs}
\usepackage{hyperref}
\usepackage{makecell}

\geometry{margin=2.5cm}

\AddToShipoutPictureBG{
  \definecolor{sidebar}{HTML}{008571}
  \begin{tikzpicture}[remember picture,overlay]
    \fill[sidebar]
    (current page.north west)
    rectangle
    ([xshift=1.2cm]current page.south west);
  \end{tikzpicture}
}

\begin{document}
\begin{titlepage}
    \thispagestyle{empty}
    \centering
    \vspace*{2cm}

    % Logos at the corners
    \begin{figure}[t]
        \centering
        \includegraphics[width=0.6\textwidth]{logos/nmbu_logo_tekst.jpeg}
    \end{figure}
    %\includegraphics[width=0.35\textwidth]{logos/eik_lab_logo.jpeg}\\[1.5cm]

    {\Huge\bfseries Project Louis}\\[0.5cm]
    {\Large\itshape Articulated Robot Arm}\\[1.5cm]
    \begin{figure}[H]
        \centering
        \includegraphics[width=.3\textwidth]{logos/eik_lab_logo.jpeg}
    \end{figure}
    {\Large\itshape Funded by Eik Lab (pretty please)}\\[1.5cm]

    \rule{0.6\textwidth}{1pt}\\[0.8cm]

    {\Large\bfseries Authors}\\[0.3cm]
    {\large Andreas Brustad \\ Johannes Standal}\\[1cm]

    %{\large\bfseries Supervisor}\\[0.3cm]
    %{\large Dr. Supervisor Name}\\[2cm]

    {}

    \vfill

    \small Eik Lab \\
    Norwegian University of Life Sciences
\end{titlepage}

% Eik lab watermark
\AddToShipoutPictureBG{%
  \AtPageUpperLeft{%
    \raisebox{-2.5cm}{%
      \hspace{1.072\textwidth}%
      \includegraphics[width=2cm]{logos/eik_lab_logo.jpeg}
    }%
  }%
}
\section{Introduction}

Our goal is to create a simple proof of concept for an articulated robot arm, with the 
ability to operate in the task space of $\mathbb{R}^3$. Utilizing iterative development strategies
we will research and learn new tools in order to deepen our understanding of robotics. 
Our aim is a robust project utilizing industry standard notation, protocols and user 
guidelines.

\begin{figure}[H]
    \centering
    \includegraphics[width=0.3\textwidth]{images/sketch3.png}
    \caption{Truly artistic sketch handpainted by Johannes with trackpad and MS Paint}
\end{figure}

Our inital articulated arm will consist of rigid bodies and rotational joints. We will 
follow TRR anatomy and will be equipped with a simple claw.
Using a transrotational joint at the base, followed by two rotational joints for our arm. The claw will
consist of a gripper mechanism attached to a rotational joint. 
For the first iteration we will use servo motors as they have the benefit of setting specific angles,
reducing need for sensors. This will somewhat limit our task space, however it will still be able to 
operate in task space in $\mathbb{R}^3$ bounded by our servo's angle limitations. 

The robot should be remote controlled by a controller with joystick / trackpad functionality as well
as buttons for functions such as closing / opening the claw. The program should also be equipped with methods for
both forward kinematics and numerical inverse kinematics. Utilizing this we can select arm positions in $\mathbb{R}^3$ space.
The claw should be self aligned to always match the xy-plane of our robots base. This will allow simple grab operations  

% Bill of materials
\section{BOM}
\begin{center}
    \bgroup
    \def\arraystretch{1.5}
    \hspace*{-1.3cm}
    \begin{tabular}{| c | c | c | c | c |}
        \hline
        Component & Function & Model & Reason & Amount \\
        \hline\hline
        \makecell{MCU} & \makecell{Provides \\ electrical \\ signals} & STM32 & \makecell{Functions as the "hub" of \\ all electricity in the project. \\ Can deliver what we need on command.} & 1 \\
        \hline
        \makecell{SBC} & \makecell{Provides \\ Computational \\ Power} & Raspberry Pi 5 & \makecell{Functions as the "hub" of \\ all complex computation \\ and mathematics \\ in the project} & 1 \\
        \hline
        Servo & Rotation & DS3218MG & Provides movement control & 3 \\
        \hline
        Servo & Grabbing & MG90S & Provides claw control & 2 \\
        \hline
        Breadboard & Wire hub & & \makecell{Provides a versatile way to \\ prototype electrical circuits} & 1 \\
        \hline
    \end{tabular}
    \egroup
\end{center}

\section{Project links}
\url{https://github.com/JohannesStandal/Eik---Project-Louis}

\end{document}