% report.tex - Frontpage for the report (place in Report/)
\documentclass[a4paper,12pt]{article}
\usepackage[utf8]{inputenc}
\usepackage[T1]{fontenc}
\usepackage{graphicx}
\usepackage{geometry}
\usepackage{parskip}
\usepackage{setspace}
\usepackage{caption}
\usepackage{subcaption}
\usepackage{float}
\usepackage{gensymb}
\usepackage{ulem}
\usepackage{amsmath}
\usepackage{amsfonts}
\usepackage{microtype}
\usepackage{geometry}
\usepackage{setspace}
\usepackage{tabularx}
\usepackage{longtable}
\usepackage{booktabs}
\usepackage{hyperref}
\usepackage{makecell}
\geometry{margin=2.5cm}

\begin{document}
\begin{titlepage}
    \thispagestyle{empty}
    \centering
    \vspace*{2cm}

    % Logos at the corners
    \begin{figure}[t]
        \begin{minipage}{.3\textwidth}
            \includegraphics{logos/eik_lab_logo.jpeg}
        \end{minipage}\hfill
        \begin{minipage}{.3\textwidth}
            \hspace*{-0.3\textwidth}
            \includegraphics{logos/nmbu_logo.png}
        \end{minipage}\hfill
    \end{figure}
    %\includegraphics[width=0.35\textwidth]{logos/eik_lab_logo.jpeg}\\[1.5cm]

    {\Huge\bfseries Project Louis}\\[0.5cm]
    {\Large\itshape Articulated Robot Arm}\\[1.5cm]
    {\Large\itshape Funded by Eik Lab (pretty please)}\\[1.5cm]

    \rule{0.6\textwidth}{1pt}\\[0.8cm]

    {\Large\bfseries Authors}\\[0.3cm]
    {\large Andreas Brustad \\ Johannes Standal}\\[1cm]

    %{\large\bfseries Supervisor}\\[0.3cm]
    %{\large Dr. Supervisor Name}\\[2cm]

    {}

    \vfill

    \small Eik Lab \\
    Norwegian University of Life Sciences
\end{titlepage}

\section{Introduction}

Our goal for the first iteration of the project is to create a functional
proof of concept for an articulated robot arm, with a task space in $\mathbb{R}^3$. We will be utilizing tools
we don't yet know how to use in order to learn. These are tools that
we plan to keep using for future iterations, meaning that keeping the first iteration
simple for learning is key for future success. Therefore, the first iteration will be
an extremely simplified model of an articulated robot arm, to the point where it might not
fit the definition. However, through future iterations and learning the complexity
of the project will also increase accordingly.
\begin{figure}[H]
    \centering
    \includegraphics[width=0.3\textwidth]{images/sketch3.png}
    \caption{Truly artistic sketch handpainted by Johannes with trackpad and MS Paint}
\end{figure}
We want to have a transrotational joint at the base, along with a rotational joint
at the start of the arm, and another at the halfway point of the arm. That way we get
orientational control about the base, and positional control of the end of the arm.
To accomplish this for the first iteration we will utilize servo motors for their
convenient blend of functionality and ease of use. We will also have a basic gripper
which will be controlled by a single servo utilizing a gear mechanism.
For actual functionality we want the arm to be able to perform basic movement controlled
by an XBox controller or similar. Control the movement with the joysticks and control the gripper
with a button.

% Bill of materials
\section{BOM}
\begin{center}
    \bgroup
    \def\arraystretch{1.5}
    \hspace*{-1.3cm}
    \begin{tabular}{| c | c | c | c | c |}
        \hline
        Component & Function & Model & Reason & Amount \\
        \hline\hline
        \makecell{MCU} & \makecell{Provides \\ electrical \\ signals} & STM32 & \makecell{Functions as the "hub" of \\ all electricity in the project. \\ Can deliver what we need on command.} & 1 \\
        \hline
        \makecell{SCB} & \makecell{Provides \\ Computational \\ Power} & Raspberry Pi 5 & \makecell{Functions as the "hub" of \\ all complex computation \\ and mathematics \\ in the project} & 1 \\
        \hline
        Servo & Rotation & DS3218MG & Provides movement control & 3 \\
        \hline
        Servo & Gripping & MG90S & Provides gripper control & 2 \\
        \hline
        Breadboard & Wire hub & & \makecell{Provides a versatile way to \\ prototype electrical circuits} & 1 \\
        \hline
    \end{tabular}
    \egroup
\end{center}

\section{Project links}
\url{https://github.com/JohannesStandal/Eik---Project-Louis}

\end{document}